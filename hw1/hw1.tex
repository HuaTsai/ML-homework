\documentclass[12pt]{article}
 
\usepackage[margin=1in]{geometry} 
\usepackage{amsmath,amsthm,amssymb}
\usepackage{enumitem}
\usepackage{listings}
\usepackage{xcolor}
\numberwithin{equation}{section}
\DeclareMathOperator{\E}{\mathbb{E}}
\definecolor{dkgreen}{rgb}{0,0.6,0}
\definecolor{gray}{rgb}{0.5,0.5,0.5}
\definecolor{mauve}{rgb}{0.58,0,0.82}
\lstset{
  language=Python,
  aboveskip=3mm,
  belowskip=3mm,
  numbers=none,
  showstringspaces=false,
  columns=flexible,
  basicstyle=\ttfamily,
  numberstyle=\color{gray},
  keywordstyle=\color{blue},
  commentstyle=\color{dkgreen},
  stringstyle=\color{mauve},
  backgroundcolor=\color{black!5}, % set backgroundcolor
}

\newenvironment{theorem}[2][Theorem]{\begin{trivlist}
\item[\hskip \labelsep {\bfseries #1}\hskip \labelsep {\bfseries #2.}]}{\end{trivlist}}
\newenvironment{lemma}[2][Lemma]{\begin{trivlist}
\item[\hskip \labelsep {\bfseries #1}\hskip \labelsep {\bfseries #2.}]}{\end{trivlist}}
\newenvironment{exercise}[2][Exercise]{\begin{trivlist}
\item[\hskip \labelsep {\bfseries #1}\hskip \labelsep {\bfseries #2.}]}{\end{trivlist}}
\newenvironment{reflection}[2][Reflection]{\begin{trivlist}
\item[\hskip \labelsep {\bfseries #1}\hskip \labelsep {\bfseries #2.}]}{\end{trivlist}}
\newenvironment{proposition}[2][Proposition]{\begin{trivlist}
\item[\hskip \labelsep {\bfseries #1}\hskip \labelsep {\bfseries #2.}]}{\end{trivlist}}
\newenvironment{corollary}[2][Corollary]{\begin{trivlist}
\item[\hskip \labelsep {\bfseries #1}\hskip \labelsep {\bfseries #2.}]}{\end{trivlist}}
 
\title{Homework \#1}
\author{Shao Hua, Huang\\
ECM9032 - Machine Learning}

\begin{document}
\maketitle
\section{Jensen's inequality (10\%)}
\begin{proof}
Use mathematical induction to prove Jensen's inequality.
  \begin{enumerate}[label=(\alph*)]
    \item for $M = 2$, the inequality \eqref{1_1} holds
      \begin{equation}\label{1_1} 
        f(\sum_{i=1}^{M}\lambda_ix_i)\leq\sum_{i=1}^{M}\lambda_if(x_i)
      \end{equation}
      which derived from (1) in problem description
    \item assume $M = n$ holds, \textit{i.e.}
      \begin{equation}\label{1_2}
        f(\sum_{i=1}^{n}\lambda_ix_i)\leq\sum_{i=1}^{n}\lambda_if(x_i) \text{ where }
        \lambda_i>0 \text{ and } \sum_{i=1}^n\lambda_i=1
      \end{equation}
    \item for $M = n + 1$,
      \begin{align}
        f(\sum_{i=1}^{n+1}\lambda_ix_i) & = f(\sum_{i=1}^{n}\lambda_ix_i+\lambda_{n+1}x_{n+1})\nonumber\\
         & = f\left[(1-\lambda_{n+1})\sum_{i=1}^{n}\frac{\lambda_ix_i}{1-\lambda_{n+1}}+\lambda_{n+1}x_{n+1}\right]\label{1_3}\\
         & \leq (1-\lambda_{n+1})f(\sum_{i=1}^n\frac{\lambda_ix_i}{1-\lambda_{n+1}})+\lambda_{n+1}f(x_{n+1})\label{1_4}\\
         & \leq (1-\lambda_{n+1})\sum_{i=1}^n\frac{\lambda_i}{1-\lambda_{n+1}}f(x_i)+\lambda_{n+1}f(x_{n+1})\label{1_5}\\
         & = \sum_{i=1}^{n}\lambda_if(x_i)+\lambda_{n+1}f(x_{n+1})\nonumber\\
         & = \sum_{i=1}^{n+1}\lambda_if(x_i)\nonumber
      \end{align}
      Use \eqref{1_1} to infer inequality from \eqref{1_3} to \eqref{1_4}\\
      Besides, because of $$\frac{\lambda_i}{1-\lambda_{n+1}}>0 \text{ and } \sum_{i=1}^n\frac{\lambda_i}{1-\lambda_{n+1}}=1$$\\
      We can infer \eqref{1_4} to \eqref{1_5} with \eqref{1_2}
    \item
      By mathematical induction and (a), (b), and (c),\\
      we can prove that $$f(\sum_{i=1}^{M}\lambda_ix_i)\leq\sum_{i=1}^{M}\lambda_if(x_i)\quad\forall M\in\mathbb{N}\wedge M>2$$
  \end{enumerate}
\end{proof}
\section{Entropy of the univariate Gaaussian (15\%)}
\begin{proof}
\begin{align}
  H[x] & = \E[-\ln(p(x))]\nonumber\\
       & = \E\left[-\ln\left(\frac{1}{(2\pi\sigma^2)^\frac{1}{2}}\exp^{-\frac{(x-\mu)^2}{2\sigma^2}}\right)\right]\label{2_1}\\
       & = \E\left[\frac{1}{2}\ln(2\pi\sigma^2)+\frac{(x-\mu)^2}{2\sigma^2}\right]\label{2_2}\\
       & = \frac{1}{2}\ln(2\pi\sigma^2)+\frac{1}{2\sigma^2}\E\left[(x-\mu)^2\right]\label{2_3}\\
       & = \frac{1}{2}\ln(2\pi\sigma^2)+\frac{1}{2\sigma^2}\int_{-\infty}^{\infty}(x-\mu)^2p(x)\mathrm{d}x\label{2_4}\\
       & = \frac{1}{2}\ln(2\pi\sigma^2)+\frac{1}{2\sigma^2}\sigma^2\label{2_5}\\
       & = \frac{1}{2}\left(1+\ln(2\pi\sigma^2)\right)\nonumber
\end{align}
  From \eqref{2_1} to \eqref{2_2}, we perform basic log operation.\\
  From \eqref{2_2} to \eqref{2_3}, we take out independent terms in expectations.\\
  To get \eqref{2_5}, we substitude equation about variance in problem to \eqref{2_3}.\\
  Therefore, we prove the equation $$H[x] = \frac{1}{2}\left(1+\ln(2\pi\sigma^2)\right)$$
\end{proof}
\section{KL divergence between two Gaussians (15\%)}
Assume both $\sigma$ and $s$ are positive.
\begin{align}
  \mathrm{KL}(p\parallel q)
    & = -\int p(x)\ln\left(\frac{q(x)}{p(x)}\right)\mathrm{d}x\nonumber\\
    & = -\int p(x)\ln\left(\frac{\frac{1}{\sqrt{2\pi s^2}}\exp^{-\frac{(x-m)^2}{2s^2}}}{\frac{1}{\sqrt{2\pi\sigma^2}}\exp^{-\frac{(x-\mu)^2}{2\sigma^2}}}\right)\mathrm{d}x\nonumber\\
    & = \int p(x)\left(\ln\left(\frac{s^2}{\sigma^2}\right)^\frac{1}{2}+\left(\frac{(x-m)^2}{2s^2}-\frac{(x-\mu)^2}{2\sigma^2}\right)\right)\mathrm{d}x\nonumber\\
    & = \ln\frac{s}{\sigma}+\frac{1}{2s^2}\int p(x)(x-m)^2\mathrm{d}x-\frac{1}{2\sigma^2}\int p(x)(x-\mu)^2\mathrm{d}x\label{3_1}
\end{align}
We calculate integral of second term in \eqref{3_1}
\begin{align}
  \int p(x)(x-m)^2\mathrm{d}x
    & = \int p(x)(x-\mu+\mu-m)^2\mathrm{d}x\nonumber\\
    & = \int (x-\mu)^2p(x)\mathrm{d}x+\int 2(x-\mu)(\mu-m)p(x)\mathrm{d}x+\int (\mu-m)^2p(x)\mathrm{d}x\nonumber\\
    & = \sigma^2+2(\mu-m)\int (x-\mu)p(x)\mathrm{d}x+(\mu-m)^2\int p(x)\mathrm{d}x\nonumber\\
    & = \sigma^2+(\mu-m)^2\label{3_2}
\end{align}
Substitude equation \eqref{3_2} to \eqref{3_1}, we get
\begin{align*}
  \mathrm{KL}(p\parallel q)
    & = \ln\frac{s}{\sigma}+\frac{1}{2s^2}\left(\sigma^2+(\mu-m)^2\right)-\frac{1}{2\sigma^2}\int p(x)(x-\mu)^2\mathrm{d}x\\
    & = \ln\frac{s}{\sigma}+\frac{1}{2s^2}\left(\sigma^2+(\mu-m)^2\right)-\frac{\sigma^2}{2\sigma^2}\\
    & = \ln\frac{s}{\sigma}+\frac{\sigma^2+(\mu-m)^2}{2s^2}-\frac{1}{2}
\end{align*}
\section{Polynomial Curve Fitting (30\%)}
Please use ipython to check my solution to this problem.\\
I use numpy, matplotlib, and pandas packages.
\begin{lstlisting}
run hw1_4.py
hw1_4_1()
hw1_4_2()
hw1_4_3()
\end{lstlisting}
\section{Polynomial Regression (30\%)}
Please use ipython to check my solution to this problem.\\
I use numpy, scipy and matplotlib packages.
\begin{lstlisting}
run hw1_5.py
hw1_5_1()
hw1_5_2()
\end{lstlisting}

\end{document}
